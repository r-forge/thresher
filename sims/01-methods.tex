\documentclass{nature}
\usepackage{graphicx}
\usepackage{cite}
\usepackage{hyperref}

\def\rcode#1{\texttt{#1}}
\def\fref#1{\textbf{Figure~\ref{#1}}}
\def\tref#1{\textbf{Table~\ref{#1}}}
\def\sref#1{\textbf{Section~\ref{#1}}}

\title{Clustering Biological Data Sets}
\author{Kevin R. Coombes$^1$, Gerard Lozanski$^2$, Steven M. Kornblau$^3$}
\date{7 January 2014}

\begin{document}
\maketitle
\begin{affiliations}
\item Department of Biomedical Informatics, The Ohio State University
  Wexner Medical Center, Columbus, OH 43210
\item Department of Pathology, The Ohio State University Wexner
  Medical Center, Columbus, OH 43210
\item Department of Bone Marrow Transplantation, The University of
  Texas M.D. Anderson Cancer Center, Houston, TX 77030
\end{affiliations}

\section{abstract}
something should go here
\subsection{Keywords}

hierarchical clustering, principal components analysis

Since the earliest days of gene expression microarrays, two-way
clustered heatmaps have been a standard feature of most papers
studying genome-wide biological datasets \cite{weinstein97,eisen98}.
Such heatmaps remain ubiquitous, in spite of numerous difficulties in
interpretation, reproducibility, and in assigning statistical
significance.

In this article, we present a comprehensive method for combining
principal comnponents analysis (PCA) with hierachical clustering to
analyze biological datasets. This method includes
\begin{itemize}
\item Statistical tools for removing ``outlier'' features (i.e.,
  genes, proteins, etc.) that do not contribute to clustering.
\item Statistical tools for determining the significant number of
  principal components and the number of feature clusters.
\item A method for automatically selecting the metric for hierarchical
  clustering features that best matches the clusters derived from PCA.
\item more cool stuff
\end{itemize}

To deermine the theoretical performance characteristics of the method,
we perform two sets of simulations.  In the first set of simulations,
each data set includes approximately 15 proteins, belonging to one or
two significant principal components (PCs) and between one and four
feaure clusters, and two outlier features. In the second set of
simulations, each data set contains approximately 100 features,
belonging to five signifcant PCs and ten feature clusters, and
approximately 30 outlier features. We then apply the method to two
actual biological datasets. The first consists of protein expression
data collected on samples from patients with acute myelogenous
leukemia (AML) using reverse phase protein lysate arrays (RPPA).  The
second consists of complex flow cytometry data from either peripheral
blood or apheresis samples of patients with a variety of conditions
and treatments.

\section{Results}

\subsection{First Simuklation}
We simulated 

\subsection{Second Simulation}

\subsection{RPPA Data From AML Patients}

\subsection{Flow Cytometry Data}

\section{Methods}
All analyses were performed using vesion 3.0.0 of the R statistical
software environment \cite{R} with version 1.0.0 of the
\texttt{Thresher} package, which we developed.

Here we describe the \texttt{Thresher} algorithms that we use to
cluster biological datasets. Throughout, we assume that previous
feature selection methods have reduced the number of features to less
than the number of independent samples in the data set.  (The feature
selection step may be statistical or biological in nature. For
example, one could use only the genes or proteins that are part of a
knwon biological pathway.)  The data from each feature are then
standardized to have mean zero and standard deviation one.

\subsection{Number of Principal Components}
To determine the number of significant PCs, we begin with the visual
Bayesian method proposed by Auer and Gervini \cite{auer-gervini}.
Briefly, they place an exponential prior on the number of signficant
components. This prior distribution depends on a hyperparameter,
$\theta \in[0, \infty]$, that governs how quickly the distribution
decays.  When $\theta=0$, the prior is flat, and the maximum a
posteriori (MAP) estimate of the number of PCs is always equal to the
number of features.  As $\theta\mapsto\infty$, the prior drops off
more rapidly, and the MAP estimate of the number of PCs will
$\mapsto0$. Auer and Gervini propose plotting the MAP estimate as a
(step) function of $\theta$, and then selecting the highest step whose
length is ``nontrivial''.

We have operationalized the final subjective step in their approach by
putting an upper bound on the largest reasonable value of $\theta$
(specifically, BLAH) and then defining ``nontrivial'' to mean at
least twice the median length of a step.

\subsection{Outlier Detection}
Having established the number $N$ of significant PCs, we work in the
principal component space containing the first $N$ PCs.  Each feature
$F$ is given a ``loading'' that describes its contribution to each PC,
and so can be represented by an $N$-dimensional vector. The length,
$||F||$, of this vector (in the sense of Euclidean distance in PC
space) summarizes the full extent of its contributions to describing
any structure present in the data. Based on the result of simulations,
we identify a feature to be an outlier if $||F|| < 0.3$.

\subsection{Number of Feature Clusters by PCA}
After removing outliers, we repeat PCA on the reduced data set, and
normalize the features in $N$-dimensional PC space to have unit length
($F_{(1)} = F/||F||$); each normalized vector records the direction of
a feature as a point on a unit hypersphere. We cluster these points
using a mixture of von Mises-Fisher distributions
\cite{banerjee05,hornik13} as implemented in version 0.1-2 of the
\texttt{movMF} package for the R statistcial software environment.

\section{Conclusions}


\bibliography{01-methods}
\bibliographystyle{nature}{}


\end{document}
